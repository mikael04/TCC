
\chapter{C\'odigos do Servidor}

\tocless\section{main\_posix.c}

\begin{lstlisting}
#include <sys/socket.h>
#include <netinet/in.h>
#include <stdio.h>
#include <stdbool.h>
#include <strings.h>
#include <inttypes.h>
/*INET*/
#include <arpa/inet.h>
/*SLEEP - TEST*/
#include <time.h>
/*Palavras (sair) */
#include <string.h>
/*Close and fork*/
#include <unistd.h>

/*SIGNAL, KILL FORK*/
#include <sys/types.h>
#include <sys/wait.h>
#include <signal.h>

#include <pthread.h>
/*srand and rand*/
#include <stdlib.h>

/*libs */
#include "coap.h"
#include "cli_main.h"


#define PORT_CLI 32015
/*#define PORT_SERV 7891 /*EMBARCADO NÃO REENVIA, CLIENTE ENVIARÀ -> não usada*/
#define TIME 0
#define DEBUG_MAIN 1 /*Debug Geral/
#define DEBUG_SAIR 1
#define DEBUG_SEND_ANOTHER_SERV 0
#define DEBUG_FORK 0
#define DEBUG_TIME 0 /*DEBUG com tempo padrão (não obtido do endpoint)*/
#define DEBUG_GET_TIME 1 /*DEBUG com tempo padrão (não obtido do endpoint)*/
#define DEBUG_SERV 1
#define CASA 1
#define MEGA 0

/*DEBUG
#define TIME_CALL_CLI_SEC 0.0166667*60*6 /*0.0166667*60 = 1 segundo*/
#define TIME_CALL_CLI_NSEC 0
#define NUM_THREADS 1

#define DEBUG_TIME_SEC 1
#define DEBUG_TIME_NSEC 0

#define PORCENTAGEM 100


void *thr_func_cli (void *arg)
{
struct timespec *call_cli = (struct timespec *) arg;
int clin = 0;
while(clin<1)
{
nanosleep(call_cli, (struct timespec *) NULL);
#if DEBUG_MAIN
printf("Child 1 pid, Chamando %d vez\n", clin);
printf("Get = %d\n", get_var_time());
printf("call_cli.tempo = %d\n", (int) call_cli->tv_sec);
#endif
main_cli();
clin++;
}
pthread_exit(NULL);
}

void *thr_func_serv_recv (void *arg)
{
	struct timespec *call_cli = (struct timespec *) arg;
	/*SLEEP - TEST*/
	#if TIME
	struct timespec req = {0};
	req.tv_sec = DEBUG_TIME_SEC;
	req.tv_nsec = DEBUG_TIME_NSEC;
	#endif
	#if DEBUG_MAIN && DEBUG_TIME
	*call_cli->tv_sec = TIME_CALL_CLI_SEC;
	*call_cli->tv_nsec = TIME_CALL_CLI_NSEC;
	#endif
	srand (time(NULL)); /*RANDOM FUNCTION*/
	int fd_client;
	
	/*Creating Bufs*/
	uint8_t buf[4096];
	uint8_t scratch_raw[4096];
	coap_rw_buffer_t scratch_buf = {scratch_raw, sizeof(scratch_raw)};
	
	#ifdef IPV6
	struct sockaddr_in6 servaddr;
	#else /* IPV6 */
	struct sockaddr_in servaddr;
	#endif /* IPV6 */
	
	#ifdef IPV6
	fd_client = socket(AF_INET6,SOCK_DGRAM,0);
	#else /* IPV6 */
	fd_client = socket(AF_INET,SOCK_DGRAM,0);
	#endif /* IPV6 */
	
	bzero(&servaddr,sizeof(servaddr));
	#ifdef IPV6
	servaddr.sin6_family = AF_INET6;
	servaddr.sin6_addr = in6addr_any;
	servaddr.sin6_port = htons(PORT_CLI);
	#else /* IPV6 */
	servaddr.sin_family = AF_INET;
	servaddr.sin_addr.s_addr = inet_addr("192.168.1.16");
	servaddr.sin_port = htons(32015);
	#endif /* IPV6 */
	bind(fd_client,(struct sockaddr *)&servaddr, sizeof(servaddr));
	
	while(1)
	{
		int n, rc;
		socklen_t len = sizeof(servaddr);
		coap_packet_t pkt;
		n = recvfrom(fd_client, buf, sizeof(buf), 0, (struct sockaddr *)&servaddr, &len);
		#if DEBUG_MAIN
		printf("Received this: ");
		coap_dump(buf, n, true);
		printf("\n");
		#endif
		
#if DEBUG_MAIN && DEBUG_SAIR
		if(0 == (strcmp((char*)buf, "sair\0")))
		{
			printf("Saindo do servidor\n");
			close(fd_client);
			return 0;            
		}
		else if (0 != (rc = coap_parse(&pkt, buf, n)))
			printf("Bad packet rc=%d\n", rc);
#else
		
		if (0 != (rc = coap_parse(&pkt, buf, n)))
			printf("Bad packet rc=%d\n", rc);
#endif
		else
		{
		
			coap_packet_t rsppkt;
#if DEBUG_MAIN
			coap_dumpPacket(&pkt);
#endif
			coap_handle_req(&scratch_buf, &pkt, &rsppkt);
			size_t rsplen = sizeof(buf);
			
			if (0 != (rc = coap_build(buf, &rsplen, &rsppkt)))
				printf("coap_build failed rc=%d\n", rc);
			else
			{
#if DEBUG_MAIN
				printf("Sending: ");
				coap_dump(buf, rsplen, true);
				printf("\n");
#endif
#if DEBUG_MAIN
				coap_dumpPacket(&rsppkt);
#endif
#if TIME
				nanosleep(&req, (struct timespec *) NULL);
#endif
#if DEBUG_TIME == 0
				call_cli->tv_sec = get_var_time();
				call_cli->tv_nsec = 0;
#endif
				if (rand()%100<PORCENTAGEM)
				{
					printf("\n\nSending\n\n");
					sendto(fd_client, buf, rsplen, 0, (struct sockaddr *)&servaddr, sizeof(servaddr));
				}
				else
				{
					printf("\n\nNot sending, simulando erro de comunicação\n\n");
				}                    
			}
		}
	}
	close(fd_client);    
	pthread_exit(NULL);
}

int main(int argc, char **argv)
{
	
	int i;
	pthread_t thr[NUM_THREADS];
	struct timespec call_cli = {0};
	
	endpoint_setup();
	
	call_cli.tv_sec = get_var_time();
	call_cli.tv_nsec = 0;
	
	int rc;
	
	printf("thread iniciando cliente \n");
	if ((rc = pthread_create(&thr[0], NULL, thr_func_cli, &call_cli)))
	{
		fprintf(stderr, "error: pthread_create, rc: %d\n", rc);
		return EXIT_FAILURE;
	}
	printf("thread iniciando servidor \n");
	if ((rc = pthread_create(&thr[1], NULL, thr_func_serv_recv, &call_cli)))
	{
		fprintf(stderr, "error: pthread_create, rc: %d\n", rc);
		return EXIT_FAILURE;
	}
	
	for (i = 0; i < NUM_THREADS; ++i)
	{
		pthread_join(thr[i], NULL);
	}
	
	return 0;
}
\end{lstlisting}

\tocless\section{endpoints.c}

\begin{lstlisting}

#include <stdbool.h>
#include <string.h>
#include <time.h>
#include "coap.h"

#include <stdlib.h>

#define DEBUG_END 1
#define TEMP_1_CHAR 0
#define TEMP_STRING 1
#define DEBUG_STRUCT_TIME 1

static char light = '0';
static char temperature[4] = "0";
static char time_freq_string[4] = "15";
var_cli variable = {0};

const uint16_t rsplen = 1500;
static char rsp[1500] = "";
void build_rsp(void);

#ifdef ARDUINO
#include "Arduino.h"
static int led = 6;
void endpoint_setup(void)
{                
	pinMode(led, OUTPUT);     
	build_rsp();
}
#else
#include <stdio.h>
void endpoint_setup(void)
{
	build_rsp();
	variable = create_var_time ();
}
#endif


/* Variável de tempo (usar no main-posix)*/
void set_var_time (short int tempo)
{
	variable.tempo = tempo;
}
short int get_var_time ()
{
	return variable.tempo;
}

var_cli create_var_time ()
{
	var_cli variable;
	variable.tempo = 5; /*Segundos, para testes*/
	return variable;
}

static const coap_endpoint_path_t path_well_known_core = {2, {".well-known", "core"}};
static int handle_get_well_known_core(coap_rw_buffer_t *scratch, const coap_packet_t *inpkt, coap_packet_t *outpkt, uint8_t id_hi, uint8_t id_lo)
{
	return coap_make_response(scratch, outpkt, (const uint8_t *)rsp, strlen(rsp), id_hi, id_lo, &inpkt->tok, COAP_RSPCODE_CONTENT, COAP_CONTENTTYPE_APPLICATION_LINKFORMAT);
}

static const coap_endpoint_path_t path_light = {1, {"light"}};
static int handle_get_light(coap_rw_buffer_t *scratch, const coap_packet_t *inpkt, coap_packet_t *outpkt, uint8_t id_hi, uint8_t id_lo)
{
	return coap_make_response(scratch, outpkt, (const uint8_t *)&light, 1, id_hi, id_lo, &inpkt->tok, COAP_RSPCODE_CONTENT, COAP_CONTENTTYPE_TEXT_PLAIN);
}

static int handle_post_light(coap_rw_buffer_t *scratch, const coap_packet_t *inpkt, coap_packet_t *outpkt, uint8_t id_hi, uint8_t id_lo)
{
	return coap_make_response(scratch, outpkt, (const uint8_t *)&light, 1, id_hi, id_lo, &inpkt->tok, COAP_RSPCODE_CONTENT, COAP_CONTENTTYPE_TEXT_PLAIN);
}

/* Receber parâmetros, inicialmente serão:
Tempo = 15m -> Para testes -> 2s
*/

static const coap_endpoint_path_t path_time_freq = {2, {"var", "time"}};
static int handle_get_time_freq(coap_rw_buffer_t *scratch, const coap_packet_t *inpkt, coap_packet_t *outpkt, uint8_t id_hi, uint8_t id_lo)
{
#if DEBUG_END
	printf("\nWell get time new method\n");
	printf("Visualizando time_string");
	printf("\nTime = %s\n", time_freq_string);
#endif
	return coap_make_response(scratch, outpkt, (const uint8_t *)&(time_freq_string), strlen(time_freq_string), id_hi, id_lo, &inpkt->tok, COAP_RSPCODE_CONTENT, COAP_CONTENTTYPE_TEXT_PLAIN);

}

static int handle_post_time_freq(coap_rw_buffer_t *scratch, const coap_packet_t *inpkt, coap_packet_t *outpkt, uint8_t id_hi, uint8_t id_lo)
{
	char *end;
#if DEBUG_END
	printf("\nWell post temperature new method?\n");
#endif
	if (inpkt -> payload.len == 0)
	{
		return coap_make_response(scratch, outpkt, NULL, 0, id_hi, id_lo, &inpkt->tok, COAP_RSPCODE_BAD_REQUEST, COAP_CONTENTTYPE_TEXT_PLAIN);
	}
	else
	{
		strncpy (time_freq_string, "00000", strlen(time_freq_string));
		strncpy (time_freq_string, (char *)inpkt->payload.p, strlen(time_freq_string));
#if DEBUG_END
		printf(" payload.len = %d\n", inpkt -> payload.len);
		strncpy (time_freq_string, "00000", strlen(time_freq_string));
		strncpy (time_freq_string, (char *)inpkt->payload.p, strlen(time_freq_string));
		printf("Time_fre_string = %s", time_freq_string);
#endif
#if DEBUG_END && DEBUG_STRUCT_TIME
		printf("1)Time_freq = %d\n", get_var_time());
#endif		
		set_var_time(strtol ((char *)inpkt->payload.p, &end, 0));
#if DEBUG_END && DEBUG_STRUCT_TIME
		printf("2)Time_freq = %d\n", get_var_time());
#endif
	}
	return coap_make_response(scratch, outpkt, (const uint8_t *)&(time_freq_string), inpkt->payload.len, id_hi, id_lo, &inpkt->tok, COAP_RSPCODE_CHANGED, COAP_CONTENTTYPE_TEXT_PLAIN);
}


static const coap_endpoint_path_t path_temperature = {2, {"var", "temperature"}};
static int handle_get_temperature(coap_rw_buffer_t *scratch, const coap_packet_t *inpkt, coap_packet_t *outpkt, uint8_t id_hi, uint8_t id_lo)
{
#if DEBUG_END
	printf("\nWell get temperature new method?\n");
	printf("Visualizando temperature");
	printf("\nTemperature = %s\nLight = %c\n", temperature, light);
#endif
	return coap_make_response(scratch, outpkt, (const uint8_t *)&(temperature), strlen(temperature), id_hi, id_lo, &inpkt->tok, COAP_RSPCODE_CONTENT, COAP_CONTENTTYPE_TEXT_PLAIN);

}

static int handle_post_temperature(coap_rw_buffer_t *scratch, const coap_packet_t *inpkt, coap_packet_t *outpkt, uint8_t id_hi, uint8_t id_lo)
{
	#if DEBUG_END
	printf("\nWell post temperature new method?\n");
	#endif
	if (inpkt -> payload.len == 0)
	{
		return coap_make_response(scratch, outpkt, NULL, 0, id_hi, id_lo, &inpkt->tok, COAP_RSPCODE_BAD_REQUEST, COAP_CONTENTTYPE_TEXT_PLAIN);
	}
	else
	{
#if DEBUG_END
		/*necessário temperature para devolver para pkt*/
		printf(" payload.len = %d\n", inpkt -> payload.len);
		printf("Temperature = %s\n", temperature);
		strncpy (temperature, "00000", strlen(temperature));
		printf("1)Temperature = %s\n lenght = %d\n", temperature, strlen(temperature));
		strncpy (temperature, (char *)inpkt -> payload.p, inpkt -> payload.len);
		printf("2)Temperature = %s\n lenght = %d\n", temperature, strlen(temperature));
#endif
		FILE * pFile;
		time_t t = time(NULL);
		struct tm tm = *localtime(&t);
		char date[10];
		snprintf(date,10, "%d",tm.tm_mday);
		pFile = fopen (date,"a+");
		#if TEMP_1_CHAR
		if (inpkt -> payload.len > 5 || inpkt -> payload.len < 1)
			return coap_make_response(scratch, outpkt, NULL, 0, id_hi, id_lo, &inpkt->tok, COAP_RSPCODE_BAD_REQUEST, COAP_CONTENTTYPE_TEXT_PLAIN);
		else if (pFile!=NULL)
		{
			fprintf (pFile, "Temperature = ");
			if (inpkt -> payload.len < 2)
			{
				fprintf (pFile, "%c", inpkt -> payload.p [0]);
			}
			else if (inpkt -> payload.len  < 3)
			{
				fprintf (pFile, "%c", inpkt -> payload.p [0]);
				fprintf (pFile, "%c", inpkt -> payload.p [1]);
			}
			else if (inpkt -> payload.len  < 4)
			{
				fprintf (pFile, "%c", inpkt -> payload.p [0]);
				fprintf (pFile, "%c", inpkt -> payload.p [1]);
				fprintf (pFile, "%c", inpkt -> payload.p [2]);
			}
			else if (inpkt -> payload.len  < 5)
			{
				fprintf (pFile, "%c", inpkt -> payload.p [0]);
				fprintf (pFile, "%c", inpkt -> payload.p [1]);
				fprintf (pFile, "%c", inpkt -> payload.p [2]);
				fprintf (pFile, "%c", inpkt -> payload.p [3]);
			}
		
			fprintf (pFile, " C  \t|   Log - Data: %d / %d Hora: %d :%d : %d\n", tm.tm_mon, tm.tm_mday, tm.tm_hour, tm.tm_min, tm.tm_sec);
			fclose (pFile);
		}
#endif
#if TEMP_STRING
		if(pFile!=NULL)
		{
	#if DEBUG_END
			printf("payload = %s\n", inpkt->payload.p);
	#endif
			fprintf(pFile, "\n%s", (char *)inpkt->payload.p);
			fclose(pFile);
		}
#endif
		else
		{
			return coap_make_response(scratch, outpkt, NULL, 0, id_hi, id_lo, &inpkt->tok, COAP_RSPCODE_BAD_REQUEST, COAP_CONTENTTYPE_TEXT_PLAIN);
		}
	}
	return coap_make_response(scratch, outpkt, (const uint8_t *)&temperature, inpkt->payload.len, id_hi, id_lo, &inpkt->tok, COAP_RSPCODE_CHANGED, COAP_CONTENTTYPE_TEXT_PLAIN);
}


static int handle_put_light(coap_rw_buffer_t *scratch, const coap_packet_t *inpkt, coap_packet_t *outpkt, uint8_t id_hi, uint8_t id_lo)
{
#if DEBUG_END	
	printf("\nWell put light\n");
#endif
	if (inpkt->payload.len == 0)
		return coap_make_response(scratch, outpkt, NULL, 0, id_hi, id_lo, &inpkt->tok, COAP_RSPCODE_BAD_REQUEST, COAP_CONTENTTYPE_TEXT_PLAIN);
	if (inpkt->payload.p[0] == '1')
	{
		light = '1';
#ifdef ARDUINO
		digitalWrite(led, HIGH);
#else
		printf("ON\n");
#endif
		return coap_make_response(scratch, outpkt, (const uint8_t *)&light, 1, id_hi, id_lo, &inpkt->tok, COAP_RSPCODE_CHANGED, COAP_CONTENTTYPE_TEXT_PLAIN);
	}
	else
	{
		light = '0';
#ifdef ARDUINO
		digitalWrite(led, LOW);
#else
		printf("OFF\n");
#endif
		return coap_make_response(scratch, outpkt, (const uint8_t *)&light, 1, id_hi, id_lo, &inpkt->tok, COAP_RSPCODE_CHANGED, COAP_CONTENTTYPE_TEXT_PLAIN);
	}
}

const coap_endpoint_t endpoints[] =
{
	{COAP_METHOD_GET, handle_get_well_known_core, &path_well_known_core, "ct=40"},
	{COAP_METHOD_GET, handle_get_light, &path_light, "ct=0"},
	{COAP_METHOD_GET, handle_get_temperature, &path_temperature, "ct=6"},    
	{COAP_METHOD_GET, handle_get_time_freq, &path_time_freq, "ct=7"},
	{COAP_METHOD_PUT, handle_put_light, &path_light, NULL},
	{COAP_METHOD_POST, handle_post_temperature, &path_temperature, "ct=8"},
	{COAP_METHOD_POST, handle_post_light, &path_light, "ct=4"}, 
	{COAP_METHOD_POST, handle_post_time_freq, &path_time_freq, "ct=9"},
	{(coap_method_t)0, NULL, NULL, NULL}
};

void build_rsp(void)
{
	uint16_t len = rsplen;
	const coap_endpoint_t *ep = endpoints;
	int i;
	
	len--; /* Null-terminated string*/
	
	while(NULL != ep->handler)
	{
		if (NULL == ep->core_attr)
		{
			ep++;
			continue;
		}
		
		if (0 < strlen(rsp))
		{
			strncat(rsp, ",", len);
			len--;
		}
		
		strncat(rsp, "<", len);
		len--;
		
		for (i = 0; i < ep->path->count; i++)
		{
			strncat(rsp, "/", len);
			len--;
			
			strncat(rsp, ep->path->elems[i], len);
			len -= strlen(ep->path->elems[i]);
		}
		
		strncat(rsp, ">;", len);
		len -= 2;
		
		strncat(rsp, ep->core_attr, len);
		len -= strlen(ep->core_attr);
		
		ep++;
	}
}




\end{lstlisting}

